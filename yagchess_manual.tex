\documentclass[a4paper]{article}
\usepackage[T1]{fontenc}
\usepackage[utf8]{inputenc}
\usepackage{lmodern}
\usepackage[english]{babel}

\newcommand{\nameprog}{Yagchess}
\newcommand{\version}{1.0}

\title{\nameprog\ manual}
\author{Valentino Esposito}
\date{}

\begin{document}
\maketitle

\tableofcontents

\section{Introduction}
\nameprog\ is a free GUI to play Chess, written in C++ for Linux systems. It can be used to play chess against other human players on the same machine.
It is designed to be light and easy to use. \nameprog\ stands for Yet Another Gui for CHESS. This is version \version.

It can be used to play chess against a chess engine, however no chess engine is provided with \nameprog. You have to obtain and install a compatible chess engine,
and set \nameprog\ to use it. Section \ref{howce} explains how to do these things.

\nameprog\ uses the \textbf{UCI protocol} to interact with the chess engine, hence any UCI-compatible chess engine should work.
You can find free chess engines on the internet, such as Stockfish or Gnuchess.

This manual explains how to install \nameprog\ in Section \ref{install}, provides a short description on how to start a game in Section \ref{quick}, describes
other useful features of the program in Sections \ref{gamefeat}, explains how to set the GUI to play against a chess engine in Section \ref{howce}. Finally a
description of the menu bar is provided in Section \ref{menu}.\\

\nameprog\ is free and distributed under the GNU General Public License (GPL). Essentially, this means that you are free
to do almost exactly what you want with the program, including distributing it among your friends, making it available for download from your web site,
selling it (either by itself or as part of some bigger software package), or using it as the starting point for a software project of your own.

The only real limitation is that whenever you distribute \nameprog\ in some way, you must always include the full source code,
or a pointer to where the source code can be found. If you make any changes to the source code, these changes must also be made available under the GPL.

For full details, read the copy of the GPL found in the file named LICENSE.txt


\section{Installation}
\label{install}
\subsection{Package contents}
Before installing check that the distribution contains all the files.
\nameprog\ version \version\ consists of the following files:

\begin{itemize}
\item README.txt, a text file for quick reference.
\item LICENSE.txt, a text file containing the GNU General Public License.
\item \textbf{src}, a subdirectory containing the source code in C++, including a Makefile that can be used to compile \nameprog.
\item \textbf{resources}, a subdirectory containing xml and css files needed to build the GUI.
\item \textbf{chess\_pieces\_icons}, a subdirectory of resources containing the 12 images of the pieces (png format).
\item \textbf{documentation}, a directory containing the manual in pdf format (the one you are reading) and the tex source of the manual.
\end{itemize}

All these files are in the main directory \textbf{yagchess1.0-linux}.


\subsection{Installation under Linux}
\nameprog\ can be installed on any Linux systems. To compile \nameprog, you need the \textbf{gnu gcc} compiler and the \textbf{gtkmm3.0} library.
The gcc compiler is already installed on most linux distributions.

If you do not have gtkmm, the best way to get it is probably to install it by using you package manager.
Open a terminal and type:

\begin{quote}
apt-get install libgtkmm-3.0-dev
\end{quote}

if you use a Debian based distribution or

\begin{quote}
yum install gtkmm30-docs
\end{quote}

if you use a Fedora/Red Hat based distribution. See also the gtkmm website.\footnote{https://www.gtkmm.org/en/download.html}
You probably need root privilege to install the packages.

You can also install gtkmm from source code, if you prefer this way follow the instruction on the gtkmm website.\\

Once you have installed gtkmm, you can compile the program. Go in the \textbf{src} directory and type:

\begin{quote}
make install
\end{quote}

That's all. The program is simple, there is no need of a configure script.

The binary file is created in the main directory, it's called \textbf{yagchess.x}


\subsection{Uninstallation}
To remove executable and object files, go in the src directory and type:

\begin{quote}
make clean
\end{quote}

This will not remove source code, documentation or resources. You can delete all these files manually.


\subsection{Installation under other systems}
\nameprog\ has been written for Linux, so it will not work under other system.

If you wish to port it under other systems, consider that the program is written following the C++14 standard. It uses only the C++14 standard library and gtkmm3.0
(and gtkmm dependencies such as libsigc++), plus few functions from the C unix library to manage interprocess comunication, included in the unistd.h and sys/wait.h headers.
There are no other dependencies.

So if you have a C++14 compiler and you manage to install gtkmm3.0 on your system, you just have to fix the interprocess comunication class in the
ipcproc.hpp and ipcproc.cpp files by using any function supported by your operative system (or see if those headers are supported by your system).


\section{Quick start}
\label{quick}
\subsection{Play a new game against another human}
When you start the program, you will see the interface with the chessboard on the left side, few buttons on the right side which provide the main functionalities,
a text area on the right side just below the buttons and another text area on the bottom.

Click on the \textbf{New} button to start a new game. The pieces will appear on the chessboard. To actually start the game, you need to press also the \textbf{Start} button.
This is needed in order to edit some options which are available only after a new game is started (such as the chess engine options, see Section \ref{howce:opt}).

To move a piece, simply click on it. The program highlight the squares where the pieces can be moved. Click on one of these squares to complete the move.
You play in front of the same machine: simply share the mouse when you have concluded your move.


\subsection{Play against a chess engine}
As said, the \nameprog\ GUI can be used to play against any UCI-compatible chess engine you may have installed on your system. See
Section \ref{howce} to know how to add a chess engine to \nameprog.

Playing against a chess engine is pretty much the same of playing against another person. You do your move and wait for the move of the chess engine.
The bottom text area shows the informations sent by the chess engine.

To play against a chess engine open the \textit{Game $\rightarrow$ Preferences} panel from the menu bar and edit the entry in the \textit{Set players} option. Select one of
the ``human vs chess engine'' options: you may choose who is white and who is black between you or the chess engine, or alternate the colours each game.

A text engine will take his time to think on the move. The button \textbf{Stop CE} can be used to stop a chess engine, and force it to play immediately.
The chess engine will play the best move it found in the given time. A click on the chessboard with the \textit{right button} of the mouse has the same effect.

You can set a maximum time for the chess engine to think: check the flag \textbf{Limit CE thinking time} in the right side of the interface and set the time
in the spin button. The chess engine will not think longer than the given time. If a time limit is not set, the chess engine may think indefinitely, until
you stop it, but see also Section \ref{time} for more details.


\section{GUI features}
\label{gamefeat}
\subsection{Save and Load the game}
\label{saveload}
\nameprog\ uses the Portable Game Notation (PGN) format. PGN is a light international notation used to store chess games on a machine.
You can learn more on PGN on the internet.\footnote{http://www.saremba.de/chessgml/standards/pgn/pgn-complete.htm}

The \textit{Game $\rightarrow$ Load PGN file} and \textit{Game $\rightarrow$ Save as PGN} panels from the menu bar allow respectively to load a game saved in PGN format
and to save the current game you are playing in a PGN file on your disk. You can use these features to save a game and continue it later, or to look at any game recorded 
in a PGN file in \nameprog.

Many tournaments record their games in PGN format, and make them available on the internet. \nameprog\ can deal with PGN files containing multiple games, however it
can not save multiple games in a single PGN file.\\

Another way to load and save the game is through the use of the Forsyth-Edwards Notation (FEN). The FEN is a representation of the chessboard in a single string of text. You
can find more on the FEN on the internet.

The \textit{Game $\rightarrow$ Build from FEN} panel from the menu bar allows you to write a FEN string in an entry. The pieces on the chessboard will be placed according
to the FEN, so that you can continue the game from this disposition. The \textit{Writing $\rightarrow$ Get FEN representation} panel
from the menu bar write the FEN representation of the current game in a label, so that you can copy it and save in a file.


\subsection{Timed game}
\label{time}
The \textit{Game $\rightarrow$ Preferences} panel from the menu bar also allows you to set a time limit. The time is shown at the top of the interface: two clocks
which count down the time during the turn of the player.

You can set between various time options, from a minimum of \textit{10 minutes} to a maximum of \textit{3 hours}. This is the time for each player. Of course there is
also the \textit{no limit} option, which is the default. Time limit is sudden death: no extra time is added during the game.

If you are playing against a chess engine, a time limit is set and the \textbf{Limit CE thinking time} flag is unchecked, the chess engine will evaluate by itself the time
it needs to think, Of course, you can always force it to move immediately by pressing the \textbf{Stop CE} button or by \textit{right clicking} on the chessboard.


\subsection{Other features}
The interface provides other useful buttons. The \textbf{Back} and \textbf{Forward} buttons can be used to move along the history of the game, reviewing the previous moves.
You can modify your previous moves if you wish, restarting the game from that point. The \textbf{Resign} button is used to surrend and close the game.

A \textit{right click} on the chessboard during the turn of the human player opens a popup menu which contains features useful to beginner players. They refers to the pieces
in the square where the right click occurred:
\begin{description}
\item[Menacing:] highlights the opponent's pieces menaced by the piece in the square, if present.
\item[Protecting:] highlights the one's pieces proteced by the piece in the square, if present.
\item[Menaced by:] highlights the opponent's pieces which are menacing the piece in the square or the free square.
\item[Protected by:] highlights the one's pieces which are protecting the piece in the square or the free square.
\end{description}


\section{How to use chess engines}
\label{howce}
\subsection{What is a chess engine}
A chess engine is a program able to play chess. Simply speaking, given a disposition of pieces on the chessboard, it can evaluate the best move for the player.
How it does this, is another matter. You can learn more on chess engines on the internet if you wish.

Usually chess engines are stand-alone projects. They focus on the ability of the program to evaluate the best move, and do not come with a graphical interface (GUI).
They use a communication protocol such as the Universal Chess Interface (UCI)\footnote{Here the specifics of the UCI protocol: http://wbec-ridderkerk.nl/html/UCIProtocol.html}
to communicate with a GUI.

\nameprog\ uses the UCI protocol to ``talk'' with the chess engine. So as long the chess engine can use the UCI protocol, it can be used with \nameprog.

Two UCI-compatible open source chess engines distributed under the GNU public license are Stockfish\footnote{https://stockfishchess.org/}
and Gnuchess\footnote{https://www.gnu.org/software/chess/}. You may start with these, if you wish, or look for others chess engines on the internet.


\subsection{Installing a chess engine}
First, you have to install a chess engine. To do this follow the instructions of the chess engine you want to install. Be sure that is an UCI-compatible chess engine,
as \nameprog\ uses the UCI protocol as explained before.


\subsection{Add a chess engine to \nameprog}
Once you have installed the chess engine, open the \textit{Game $\rightarrow$ Preferences} panel from the menu bar.

In the \textit{New chess engine name} entry put the name of the chess engine. It has not to be the real name of the chess engine,
it is just a name you use to identify this chess engine inside \nameprog.

In the \textit{New chess engine path} entry put the full path of the chess engine. You need the path of the binary file representing the chess engine program.
You can write it manually, or use the button \textbf{Select CE}. This button opens a dialog which allows you to navigate in your system and find the file.
Select the correct file and press the button \textbf{Select}: this will close the dialog and copy the path in the \textit{New chess engine path} entry.

Suppose you have installed Gnuchess version 6.2.4, the string in the \textit{New chess engine path} entry should look like:

\begin{quote}
/home/user/gnuchess-6.2.4/src/gnuchessu
\end{quote}

The full path of course may be different than this one, depending on where you installed the chess engine.

Now press the button \textbf{Add to the list}. Now the chess engine is added to the \textit{Chess Engine List} on top of the panel.
Be careful: invalid paths are not recognized here, so if there is a typo or an error in the path, you will discover it only when you start the game.\\

Once the chess engine is in the list, you have to assign it to a player (white or black). \textit{Right click} on the chess engine line in the Chess Engine List
and from the popup menu choose \textit{Set for White} if you want to assing this chess engine to the white player, or choose \textit{Set for Black}
if you want to assing it to the black player.

This configuration gives you complete control on which colour should be moved by the chess engine, and which chess engine must be used in case you have several chess engines
in your list. You may set different chess engines for white and black, to play agains different ``opponents'' when you are white or black. You may even set a game
between two chess engines and see which is stronger. Of course you may also use the same engine for white and black, so you can face the same ``opponent'' when
you are the white player and the black player.

Of course you may open the \textit{Game $\rightarrow$ Preferences} panel any time and change the chess engine you want to use, simply by \textit{right clicking}
on the new chess engine and set it for a colour. The previous association will be overwritten.

Now when you select ``human vs chess engine'' in the \textit{Game $\rightarrow$ Preferences} panel, \nameprog\ will use the selected chess engine.\\

Do not forget to press the \textbf{Save current settings} button or the next time you start \nameprog, you will have to repeat the whole procedure.


\subsection{Use a chess engine to analyse the game}
\label{howce:analys}
By \textit{right clicking} a chess engine on the \textit{Chess Engine List} in the \textit{Game $\rightarrow$ Preferences} panel, you may also choose
\textit{Set for Analyser}. The chess engine will be started when you press the \textbf{Analyse move} button in the interface. You can use this feature to let the chess engine
suggest you a move or let it analyse any move you wish to do.

The Analysis panel opened by \textbf{Analyse move} allows you to set restrictions in searching the best move. You can search the bestmove between a list of moves. Write the list
in the entry \textit{Restrict search}: the moves should be in long algebraic notation (e.g. e2e4) and separated by commas. The chess engine returns the best move
between them.

Other options allow you to limit the time of the search, the depth, the number or nodes, or to search if a checkmate could be obtained in a given number of moves.

The results of the search are given in the text area below the panel.\\

You can also edit the full set of options of the chess engine you are using as analyser by clicking on the \textbf{Chess Engine Options} in the Analysis panel: doing this
the option panel is open (see Section \ref{howce:opt}).


\subsection{Chess engine options}
\label{howce:opt}
Many chess engines have options or parameters which can be modified by the user (for example, the level of the chess engine). For a complete description of these options,
please refer to the manual of the chess engine you are using.

The UCI protocol allows \nameprog\ to ask the chess engine for all its options so that they can be configured by the user. Once you have added a chess engine to the list, 
you can set its options in two ways.

You can \textit{right click} on the chess engine in the Chess Engine List, and from the popup menu choose \textit{Options}. The option panel will appear, it shows all the options
of the chess engine and their current values. From it you can modify the option values from this panel. Once you are satisfied, press the \textbf{Save changes} button to confirm
your settings. Your values are saved in a configuration file of \nameprog, so each time you use this chess engine, it will automatically use your values.

The second way is to start the game. Start a new game with the \textbf{New} button. The chess engine will be started. Now you can open the
\textit{Chess Engine $\rightarrow$ White Options} or \textit{Chess Engine $\rightarrow$ Black Options} panel, depending on which player is the chess engine. The option panel appears.
You can edit the parameters and then press the button \textbf{Apply changes} to confirm your settings. Now the chess engine will use these values, but will not remember them
for the next game unless you click also on the \textbf{Save Changes} button. your values are saved in a configuration file of \nameprog, so each time you use this chess engine,
it will automatically use your values. You can always go back to the default values pressing the \textbf{Reset default values} button. Once you have set the chess engine
parameters, press the \textbf{Start} button to start the game.


\section{The menu bar}
\label{menu}
The menu bar contains all the actions and options you can do or set in \nameprog. Here is a short description of each one of them, and
in square brackets the keyboard shortcut you can use to activate the corresponding action.

\begin{itemize}
\item \textbf{Game}: contains actions related to the game management:
\begin{itemize}
\item \textbf{New}: prepares a new game, same as the button \textbf{New}. \mbox{[Ctrl + N]}
\item \textbf{Load PGN file}: loads a game from a PGN file (see Section \ref{saveload}). \mbox{[Ctrl + L]}
\item \textbf{Build from FEN}: builds a game from a valid FEN string you can provide (see Section \ref{saveload}). \mbox{[Ctrl + D]}
\item \textbf{Save as PGN}: saves the current game in PGN format (see Section \ref{saveload}). \mbox{[Ctrl + S]}
\item \textbf{Close}: closes the current game. \mbox{[Ctrl + C]}
\item \textbf{Quit}: closes the current game and the program. \mbox{[Ctrl + Q]}
\item \textbf{Preferences}: opens the preferences panel (see Sections \ref{gamefeat}, \ref{howce}). \mbox{[Ctrl + P]}
\end{itemize}

\item \textbf{Actions}: contains actions related to the game itself and its development:
\begin{itemize}
\item \textbf{Start}: starts a new game, same as the button \textbf{Start}. \mbox{[Ctrl + E]}
\item \textbf{Back}: goes back one move, same as the button \textbf{Back}. \mbox{[Ctrl + K]}
\item \textbf{Back All}: goes back all moves (to the beginning). \mbox{[Ctrl + Z]}
\item \textbf{Forward}: goes forward one move (if you went back before), same as the button \textbf{Forward}. \mbox{[Ctrl + F]}
\item \textbf{Forward All}: goes forward all moves (to the last move played). \mbox{[Ctrl + X]}
\item \textbf{Resign}: resigns and closes the game, same as the button \textbf{Resign}. \mbox{[Ctrl + R]}
\end{itemize}

\item \textbf{Chess Engine}: contains actions directly related to the chess engines:
\begin{itemize}
\item \textbf{White Options}: sets options of the chess engine associated with white (see Section \ref{howce:opt}). \mbox{[Ctrl + W]}
\item \textbf{Black Options}: sets options of the chess engine associated with black (see Section \ref{howce:opt}). \mbox{[Ctrl + B]}
\item \textbf{Stop Chess Engine}: stops a chess engine which is thinking and forces it to move, same as the button \textbf{Stop CE} or a \textit{right click}
of the mouse on the chessboard when the chess engine is thinking. \mbox{[Ctrl + O]}
\item \textbf{Analyse Move}: starts the chess engine to analyse a game (see Section \ref{howce:analys}), same as the button \textbf{Analyse Move}. \mbox{[Ctrl + A]}
\end{itemize}

\item \textbf{Writing}: contains actions which can be used to write the current game on the disk in text files (but not in PGN format):
\begin{itemize}
\item \textbf{Write short algebraic notation}: writes a text file of the current game in short algebraic notation. \mbox{[Ctrl + I]}
\item \textbf{Write long algebraic notation}: writes a text file of the current game in long algebraic notation. \mbox{[Ctrl + J]}
\item \textbf{Show short algebraic notation}: shows the current game in short algebraic notation in a popup label. \mbox{[Ctrl + T]}
\item \textbf{Show long algebraic notation}: shows the current game in long algebraic notation in a popup label. \mbox{[Ctrl + U]}
\item \textbf{Get FEN representation}: shows the FEN representation of the chessboard in a popup label (see Section \ref{saveload}). \mbox{[Ctrl + Y]}
\end{itemize}

\item \textbf{About}: contains actions which display informations on \nameprog:
\begin{itemize}
\item \textbf{About}: shows very brief informations on the game and the license. \mbox{[Ctrl + M]}
\item \textbf{Help}: shows a panel with a quick help and reference. \mbox{[Ctrl + H]}
\end{itemize}

\end{itemize}

\end{document}
